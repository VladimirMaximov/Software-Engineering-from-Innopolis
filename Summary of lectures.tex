\documentclass[12pt, a4paper]{article}
\usepackage[utf8]{inputenc}
\usepackage[T2A]{fontenc}
\usepackage[english,russian]{babel}
\usepackage[usenames]{color}
\usepackage{hyperref}

\title{Software Engineering from Innopolis}
\author{Vladimir Maximov}
\date{\today}

\begin{document}

\maketitle

\section{Итеритуемые объекты в Python}

\subsection{Цель}

Изучить коллекции в Python и работу с ними.

\subsection{Задачи}

\begin{itemize}
    \item Ознакомиться со списками в Python.
    \item Рассмотреть популярные операции, используемые
    над списками slice, filter, map, reduce.
    \item Ознакомиться с кортежем в Python.
    \item Разобрать различные подходы создания коллекций.
    \item Ознакомиться с множествами в Python.
    \item Понять, чем руководствоваться при выборе 
    коллекции для хранения объектов.
\end{itemize}

\subsection{Список}

Список - упорядоченная и изменяемая коллекция объектов:

\begin{verbatim}
1   my_list = [1, 2, 3, None, [0], "word"]
\end{verbatim}

Слайс (срез) списка - создание нового списка, 
используя элементы уже существующего:

\begin{verbatim}
1   my_list[2:4:1]
\end{verbatim}

Берем со 2 элемента включительно по 4 элемент 
невключительно с шагом 1.

% Это пропуск строки
\vspace{1em}

Краткая форма создания списка называется
List comprehention (генератор списка):
\begin{verbatim}
1   [x for x in range(10)]
\end{verbatim}

\subsection{Кортеж}

Кортеж - упорядоченная и неизменяемая коллекция:
\begin{verbatim}
1   (1, 2, 3, 4, 5)
\end{verbatim}

\subsection{Множество}

Множество - неупоярдоченная коллекция значений, 
в которой не допускаются повторения
и не может содержать не хешируемых объектов:
\begin{verbatim}
1   {1, 2, 3, "some string"}
\end{verbatim}

\subsection{Словари}

Словарь - изменяемая коллекция объектов, которые 
обладают ключевыми словами. В Python словарь можно
описать указанием ключей и значений элементов
или генерирующим выражением:
\begin{verbatim}
1   {"key1" : "value1",
2   "key2" : "value2",
3   "key3" : "value3"}
\end{verbatim} 

\subsection{Использование памяти}

Для хранения коллекций выделяется чуть больше памяти,
чем фактически необходимо. Делается это для того, 
чтобы интерпретатор Python при добавлении элементов
выделял память реже.

\vspace{1em}

В Python в коллекции можно хранить разные объекты, 
каждый из которых может быть непредвиданного размера.
Поэтому в Python в списке хранятся указатели на нужные объекты.

\subsection{Itertools}

itertools - модуль, который предоставляет различные функции 
для работы с итерируемыми объектами. Его можно использовать
для упрощения записи операций над итерируемыми объектами.
Примеры методов:

\vspace{1em}

itertools.combinations - метод для поиска подмножеств
итерируемого объекта, возвращает генератор: 
\begin{verbatim}
1   import itertools    
2   itertools.combinations("ABCD", 2)
3   list(itertools.combinations("ABCD", 2))
4   # Вывод: [(A, B), (A, C), (A, D), (B, C), (B, D), (C, D)]
\end{verbatim}

\vspace{1em}

itertools.compress - метод, который выбирает из исходного
итерируемого объекта элементы, согласно селектору (маске):
\begin{verbatim}
1   import itertools    
2   itertools.compress([1,2,3,4], [1,0,1,0])
3   list(itertools.compress([1,2,3,4], [1,0,1,0]))
4   # Вывод: [1,3]
\end{verbatim}
    
\vspace{1em}

\section{Функции над итерируемыми объектами}

\subsection{Цель}

Ознакомиться с тем, какие операции могут быть сделаны
на итерируемых объектах и что можно представить в качестве
итерируемого объекта.

\subsection{Задачи}

\begin{itemize}
    \item Ознакомиться как еще можно работать со списками
    \item На примерах понять, какие объекты можно представлять
    в качестве итерируемого объекта.
\end{itemize}

\subsection{Itertools (продолжение)}

Если стандартного набора библиотеки itertools не хватает,
то можно использовать сторонний пакет - 
\href{https://more-itertools.readthedocs.io/en/stable/api.html}
{\textcolor{blue}{more\textunderscore itertools}}, он расширяет
возможности работы с итерируемыми объектами. Примеры функций:

\vspace{1em}

Функция chunked:
\begin{verbatim}
1   from more_itertools import chunked
2   iterable = [0, 1, 2, 3, 4, 5, 6, 7, 8]
3   list(chunked(iterable, 3))
4   # Вывод: [[0, 1, 2], [3, 4, 5], [6, 7, 8]]
\end{verbatim}

Функция flatten:
\begin{verbatim}
1   from more_itertools import flatten
2   iterable = [(0, 1), (2, 3)]
3   list(flatten(iterable))
4   # Вывод: [0, 1, 2, 3]
\end{verbatim}

Функция split\textunderscore at:
\begin{verbatim}
1   from more_itertools import split_at
2   list(split_at('abcdcba', lambda x: x == 'b'))
3   # Вывод: [['a'], ['c', 'd', 'c'], ['a']]
\end{verbatim}

Функция transpose:
\begin{verbatim}
1   from more_itertools import transpose
2   list(transpose([(1, 2, 3), (11, 22, 33)]))
3   # Вывод: [['a'], ['c', 'd', 'c'], ['a']]
\end{verbatim}

Функция windowed:
\begin{verbatim}
1   from more_itertools import windowed
2   all_windows = windowed([1, 2, 3, 4, 5], 3)
3   list(all_windows)
4   # Вывод: [(1, 2, 3), (2, 3, 4), (3, 4, 5)]
\end{verbatim}

\subsection{Filter, Map, Reduce}

Иногда для целесообразного использования памяти лучше 
написать часть программы в функциональном стиле. 
Особенно это применимо когда нужно пройтись по большому 
количеству элементов. Для этого в python существуют встроенные
функции. Их использование будет рассмотрено на простых примерах.

\subsubsection{Filter}

Filter - функция, которая позволяет выбрать из любого 
итерируемого объекта элементы удовлетворяющие условию. 
В качестве аргумента принимает функцию или лямбда-выражение
и список, из которого отфильтрует значения:

\begin{verbatim}
1   items = [-5,-4,-3,-2,-1,0,1,2,3,4,5]
2   def my_filter_expr(item):
3       return item > 0
4   positive_items = tuple(filter(my_filter_expr, items))
5   print(positive_items)
6   # Вывод: (1, 2, 3, 4, 5)
\end{verbatim}

Важно понимать что получившийс объект не хранит в себе все 
элементы получившегося списка. Он хранит информацию о том 
как можно получить каждый из последующих элементов списка. 
Важно понимать как это работает чтобы в дальнейшем не попадать
на ошибки. Самый близкий родственный объект к получившемуся 
фильтру - итератор. Итератор это специальный объект в python 
который выдает по одном элементу и по нему можно пройтись 
только один раз.

\subsubsection{Map}

Map - функция, которая позволяет применить функцию ко всему 
списку значений. В качестве аргумента принимает функцию или 
лямбда-выражение и список, к элементам которого будет 
применена функция:

\begin{verbatim}
1   list(map(lambda a: a[0]**a[1], [(0, 2), (1, 2), (2, 2)]))
2   # Вывод: [0, 1, 4]
\end{verbatim}

\subsubsection{Reduce}

Reduce - функция, применяющая другую функцию к 
последовательным парам значений в списке, аналог функции
Fold в Wolfram Mathematica:

\begin{verbatim}
1   from functools import reduce
2   many_items = [1,2,3,4]
3   product = reduce(lambda a, b: a * b, many_items)
4   # Вывод: 24
\end{verbatim}

В начале в качестве переменной $a$ функция берет значение 1,
в качестве переменной $b$ - 2, на втором шаге переменная $a$ - 
результат функции на предыдущей итерации, $b$ - 3 и т.д. У
функции есть третий необязательный аргумент - начальное значение.

\subsection{Метод скользящего окна}

Метод скользящего окна представляет собой процесс, при котором 
окно фиксированного размера последовательно перемещается по 
набору данных. Значения внутри окна анализируются и 
обрабатываются для получения новых результатов, например, 
вычисления среднего или медианного значения.

\subsection{Операции из линейной алгебры}

Линейная алгебра это раздел математики, основными конструкциями
в которой являются списочный типы данных. Матрицы, векторы, 
тензоры - это все понятия из линейной алгебры.
Самые базовые математические операции в линейной алгебре - это 
операции на этих структурах определенных в линейном 
пространстве - транспонирование, перемножение матриц и 
векторов, нахождение определителей, решение системы 
линейныхуравнений. Пример перемнжения матриц в помощью
more\textunderscore itertools:

\begin{verbatim}
1   from more_itertools import matmul
2   matrix1 = [
3   [1,2,3],
4   [4,5,6],
5   ]
6   matrix2 = [
7   [7, 8],
8   [9, 10],
9   [11, 12],
10  ]
11  list(matmul(matrix1, matrix2))
12  # Вывод: [[58, 64], [139, 154]]
\end{verbatim}

\newpage

\subsection{Линейная алгебра с numpy}

numpy - библиотека, которая заточена под выполнение 
операций на матрицах. Основные фукнции можно посмотреть по 
\href{https://github.com/KeithGalli/NumPy/blob/master/
NumPy%20Tutorial.ipynb}{\textcolor{blue}{ссылке}}. 
Пример перемножения матриц с помощью numpy:

\begin{verbatim}
1   import numpy as np
2   matrix1 = np.array([
3   [1,2,3],
4   [4,5,6],
5   ])
6   matrix2 = np.array([
7   [7, 8],
8   [9, 10],
9   [11, 12],
10  ])
11  matrix1 @ matrix2
12  # Вывод: array([[58, 64], [139, 154]])
\end{verbatim}

\end{document}