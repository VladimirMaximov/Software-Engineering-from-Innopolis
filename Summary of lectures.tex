\documentclass[12pt, a4paper]{article}
\usepackage[utf8]{inputenc}
\usepackage[T2A]{fontenc}
\usepackage[english,russian]{babel}

\title{Software Engineering from Innopolis}
\author{Vladimir Maximov}
\date{\today}

\begin{document}

\maketitle

\section{Итеритуемые объекты в Python}

\subsection{Цель}

Изучить коллекции в Python и работу с ними.

\subsection{Задачи}

\begin{itemize}
    \item Ознакомиться со списками в Python.
    \item Рассмотреть популярные операции, используемые
    над списками slice, filter, map, reduce.
    \item Ознакомиться с кортежем в Python.
    \item Разобрать различные подходы создания коллекций.
    \item Ознакомиться с множествами в Python.
    \item Понять, чем руководствоваться при выборе 
    коллекции для хранения объектов.
\end{itemize}

\subsection{Список}

Список - упорядоченная и изменяемая коллекция объектов:

\begin{verbatim}
1   my_list = [1, 2, 3, None, [0], "word"]
\end{verbatim}

Слайс (срез) списка - создание нового списка, 
используя элементы уже существующего:

\begin{verbatim}
1   my_list[2:4:1]
\end{verbatim}

Берем со 2 элемента включительно по 4 элемент 
невключительно с шагом 1.

% Это пропуск строки
\vspace{1em}

Краткая форма создания списка называется
List comprehention (генератор списка):
\begin{verbatim}
1   [x for x in range(10)]
\end{verbatim}

\subsection{Кортеж}

Кортеж - упорядоченная и неизменяемая коллекция:
\begin{verbatim}
1   (1, 2, 3, 4, 5)
\end{verbatim}

\subsection{Множество}

Множество - неупоярдоченная коллекция значений, 
в которой не допускаются повторения
и не может содержать не хешируемых объектов:
\begin{verbatim}
1   {1, 2, 3, "some string"}
\end{verbatim}

\subsection{Словари}

Словарь - изменяемая коллекция объектов, которые 
обладают ключевыми словами. В Python словарь можно
описать указанием ключей и значений элементов
или генерирующим выражением:
\begin{verbatim}
1   {"key1" : "value1",
2   "key2" : "value2",
3   "key3" : "value3"}
\end{verbatim} 

\subsection{Использование памяти}

Для хранения коллекций выделяется чуть больше памяти,
чем фактически необходимо. Делается это для того, 
чтобы интерпретатор Python при добавлении элементов
выделял память реже.

\vspace{1em}

В Python в коллекции можно хранить разные объекты, 
каждый из которых может быть непредвиданного размера.
Поэтому в Python в списке хранятся указатели на нужные объекты.

\subsection{Itertools}

itertools - модуль, который предоставляет различные функции 
для работы с итерируемыми объектами. Его можно использовать
для упрощения записи операций над итерируемыми объектами.
Примеры методов:

\vspace{1em}

itertools.combinations - метод для поиска подмножеств
итерируемого объекта, возвращает генератор: 
\begin{verbatim}
1   import itertools    
2   itertools.combinations("ABCD", 2)
3   print(list(itertools.combinations("ABCD", 2)))
4   # Вывод: [(A, B), (A, C), (A, D), (B, C), (B, D), (C, D)]
\end{verbatim}

\vspace{1em}

itertools.compress - метод, который выбирает из исходного
итерируемого объекта элементы, согласно селектору (маске):
\begin{verbatim}
1   import itertools    
2   itertools.compress([1,2,3,4], [1,0,1,0])
3   print(list(itertools.compress([1,2,3,4], [1,0,1,0])))
4   # Вывод: [1,3]
\end{verbatim}
    
\vspace{1em}


\end{document}